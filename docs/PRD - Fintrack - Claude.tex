\documentclass[11pt,a4paper]{article}
\usepackage[utf8]{inputenc}
\usepackage[spanish]{babel}
\usepackage{geometry}
\usepackage{enumitem}
\usepackage{booktabs}
\usepackage{hyperref}

\geometry{margin=2.5cm}

\title{\textbf{Product Requirements Document (PRD)}\\
\large Fintrack - Aplicación de Control de Finanzas Personales}
\author{Análisis de Negocios}
\date{\today}

\begin{document}

\maketitle
\tableofcontents
\newpage

\section{Resumen}

Fintrack es una Progressive Web Application (PWA) diseñada para ofrecer un control financiero integral y estructurado a usuarios que buscan gestionar sus finanzas personales de manera eficiente. La aplicación es instalable desde navegadores en dispositivos iOS y Android, permitiendo el registro de ingresos y egresos mediante múltiples canales: entrada de texto, escaneo de recibos con cámara, registro manual categorizado y detección automática desde correos electrónicos.

El alcance de Fintrack incluye la gestión completa de activos, pasivos, presupuestos configurables (diarios, semanales, mensuales, anuales) y seguimiento de metas financieras. Adicionalmente, incorpora un módulo de Insights Financieros con estadísticas avanzadas, ratios como flujo de caja, y recomendaciones personalizadas generadas mediante inteligencia artificial. El objetivo principal es empoderar al usuario con visibilidad completa de su salud financiera y facilitarle la toma de decisiones informadas.

\section{Objetivos}

\begin{enumerate}[label=\textbf{O\arabic*.},leftmargin=*]
    \item \textbf{Facilitar el registro de movimientos financieros} mediante múltiples métodos (texto, OCR de recibos, manual, email) para maximizar la adopción y uso continuo.
    \item \textbf{Proporcionar visibilidad integral} sobre ingresos, egresos, activos y pasivos del usuario en tiempo real.
    \item \textbf{Permitir planificación financiera} a través de presupuestos configurables en diferentes horizontes temporales.
    \item \textbf{Impulsar el cumplimiento de metas financieras} mediante seguimiento visual y notificaciones de progreso.
    \item \textbf{Generar insights accionables} utilizando análisis de datos e inteligencia artificial para mejorar hábitos financieros.
    \item \textbf{Garantizar accesibilidad multiplataforma} mediante PWA instalable en iOS y Android sin fricción.
    \item \textbf{Minimizar costos operativos} seleccionando infraestructura serverless y servicios de IA económicos.
\end{enumerate}

\section{Funcionalidades Atómicas}

Las funcionalidades se desglosan en módulos implementables de forma independiente:

\subsection{Autenticación y Gestión de Usuario}

\begin{enumerate}[label=\textbf{F\arabic*.},leftmargin=*]
    \item Registro de usuario con email y contraseña.
    \item Login con autenticación OAuth (Google, Apple).
    \item Recuperación de contraseña mediante email.
    \item Gestión de perfil de usuario (nombre, foto, preferencias).
    \item Configuración de moneda predeterminada y zona horaria.
\end{enumerate}

\subsection{Registro de Movimientos Financieros}

\begin{enumerate}[label=\textbf{F\arabic*.},leftmargin=*,resume]
    \item Registro de ingreso mediante entrada de texto en lenguaje natural (procesado por IA).
    \item Registro de egreso mediante entrada de texto en lenguaje natural.
    \item Escaneo de recibos físicos utilizando cámara del dispositivo.
    \item Extracción de datos de recibos mediante OCR (fecha, monto, comercio, categoría sugerida).
    \item Registro manual de movimiento con campos: tipo (ingreso/egreso), monto, fecha, categoría, descripción, método de pago.
    \item Edición y eliminación de movimientos registrados.
    \item Adjuntar imagen del recibo a un movimiento existente.
\end{enumerate}

\subsection{Sincronización con Email}

\begin{enumerate}[label=\textbf{F\arabic*.},leftmargin=*,resume]
    \item Configuración de integración con cuenta de Gmail (OAuth).
    \item Programación de escaneo automático diario del buzón de correo.
    \item Detección de emails de notificaciones bancarias de ingresos y egresos.
    \item Extracción de información financiera de emails (monto, fecha, entidad, categoría sugerida).
    \item Cola de movimientos detectados pendientes de aprobación del usuario.
    \item Aprobación o rechazo individual de movimientos detectados.
    \item Aprobación masiva de movimientos con filtros.
\end{enumerate}

\subsection{Gestión de Categorías}

\begin{enumerate}[label=\textbf{F\arabic*.},leftmargin=*,resume]
    \item Listado de categorías predeterminadas para ingresos (salario, inversiones, otros).
    \item Listado de categorías predeterminadas para egresos (alimentación, transporte, vivienda, entretenimiento, salud, educación, otros).
    \item Creación de categorías personalizadas por el usuario.
    \item Edición y eliminación de categorías personalizadas.
    \item Asignación de iconos y colores a categorías.
\end{enumerate}

\subsection{Gestión de Activos y Pasivos}

\begin{enumerate}[label=\textbf{F\arabic*.},leftmargin=*,resume]
    \item Registro de activos (cuentas bancarias, inversiones, propiedades, otros).
    \item Registro de pasivos (préstamos, tarjetas de crédito, deudas).
    \item Actualización manual del valor de activos y saldo de pasivos.
    \item Visualización de patrimonio neto (activos - pasivos).
    \item Historial de evolución del patrimonio neto.
\end{enumerate}

\subsection{Presupuestos}

\begin{enumerate}[label=\textbf{F\arabic*.},leftmargin=*,resume]
    \item Creación de presupuesto diario por categoría o total.
    \item Creación de presupuesto semanal por categoría o total.
    \item Creación de presupuesto mensual por categoría o total.
    \item Creación de presupuesto anual por categoría o total.
    \item Seguimiento en tiempo real del gasto vs presupuesto.
    \item Alertas cuando se alcanza el 80\% y 100\% del presupuesto.
    \item Visualización gráfica de consumo de presupuesto.
\end{enumerate}

\subsection{Metas Financieras}

\begin{enumerate}[label=\textbf{F\arabic*.},leftmargin=*,resume]
    \item Creación de meta financiera (nombre, monto objetivo, fecha límite).
    \item Asignación de ahorros periódicos a una meta.
    \item Seguimiento visual del progreso hacia la meta.
    \item Notificaciones de hitos alcanzados (25\%, 50\%, 75\%, 100\%).
    \item Proyección de cumplimiento basada en ritmo de ahorro actual.
\end{enumerate}

\subsection{Insights Financieros}

\begin{enumerate}[label=\textbf{F\arabic*.},leftmargin=*,resume]
    \item Dashboard con resumen financiero mensual (ingresos, egresos, ahorro).
    \item Gráficos de distribución de gastos por categoría (pie chart).
    \item Gráficos de evolución temporal de ingresos y egresos (line chart).
    \item Cálculo de flujo de caja mensual y acumulado.
    \item Cálculo de tasa de ahorro mensual (ahorro/ingresos).
    \item Cálculo de ratio de endeudamiento (pasivos/activos).
    \item Detección de patrones de gasto recurrente mediante IA.
    \item Generación de recomendaciones personalizadas basadas en comportamiento financiero (reducir gastos en categoría X, aumentar ahorro, etc.).
    \item Predicción de gastos futuros basada en histórico.
    \item Exportación de reportes financieros en PDF.
\end{enumerate}

\subsection{PWA y Experiencia de Usuario}

\begin{enumerate}[label=\textbf{F\arabic*.},leftmargin=*,resume]
    \item Instalación de PWA desde navegador en iOS.
    \item Instalación de PWA desde navegador en Android.
    \item Funcionamiento offline con sincronización posterior.
    \item Notificaciones push para alertas de presupuesto y metas.
    \item Interfaz responsive adaptada a móvil y desktop.
    \item Modo oscuro y claro configurables.
    \item Onboarding interactivo para nuevos usuarios.
\end{enumerate}

\section{Stack Tecnológico}

La selección del stack prioriza la minimización de costos, escalabilidad y facilidad de desarrollo:

\subsection{Frontend}

\begin{itemize}
    \item \textbf{Framework}: Next.js 14+ con App Router (React 18+)
    \item \textbf{Lenguaje}: TypeScript para type safety
    \item \textbf{PWA}: next-pwa plugin para configuración de Service Worker y manifest
    \item \textbf{Estilos}: Tailwind CSS para diseño responsive rápido
    \item \textbf{Gráficos}: Recharts o Chart.js para visualizaciones
    \item \textbf{Gestión de Estado}: Zustand o React Context API (para mantener simplicidad)
    \item \textbf{Formularios}: React Hook Form + Zod para validación
\end{itemize}

\textit{Justificación}: Next.js permite despliegue gratuito en Vercel con excelente rendimiento. PWA nativa con soporte offline. TypeScript reduce bugs en producción.

\subsection{Backend y Base de Datos}

\begin{itemize}
    \item \textbf{BaaS}: Supabase (plan gratuito: 500MB database, 2GB bandwidth, 50MB file storage)
    \item \textbf{Base de Datos}: PostgreSQL (incluido en Supabase)
    \item \textbf{Autenticación}: Supabase Auth (OAuth con Google/Apple incluido)
    \item \textbf{Storage}: Supabase Storage para imágenes de recibos
    \item \textbf{Edge Functions}: Supabase Edge Functions (Deno) para lógica serverless
    \item \textbf{Real-time}: Supabase Realtime para sincronización instantánea
\end{itemize}

\textit{Justificación}: Supabase ofrece tier gratuito generoso, elimina necesidad de backend custom, incluye autenticación y storage. PostgreSQL permite queries complejas para analytics. Edge Functions evitan costos de servidor tradicional.

\subsection{API de Inteligencia Artificial}

\begin{itemize}
    \item \textbf{LLM}: Google Gemini 2.0 Flash (gratuito hasta 1500 requests/día)
    \item \textbf{OCR}: Google Cloud Vision API (1000 unidades/mes gratuitas) o Tesseract.js (gratuito, cliente)
    \item \textbf{Procesamiento NLP}: Gemini para extracción de intención en texto natural
\end{itemize}

\textit{Justificación}: Gemini 2.0 Flash ofrece tier gratuito suficiente para MVP y costo muy bajo en producción (\$0.075/1M tokens input). Vision API gratuita para OCR básico. Tesseract.js elimina costos API pero con menor precisión.

\subsection{Integraciones Externas}

\begin{itemize}
    \item \textbf{Email}: Gmail API (OAuth 2.0) para lectura de correos
    \item \textbf{Notificaciones Push}: Firebase Cloud Messaging (gratuito)
    \item \textbf{Analytics}: Vercel Analytics (incluido en despliegue)
\end{itemize}

\subsection{DevOps y Hosting}

\begin{itemize}
    \item \textbf{Hosting Frontend}: Vercel (gratuito para proyectos personales, 100GB bandwidth)
    \item \textbf{CI/CD}: GitHub Actions (2000 min/mes gratuitas)
    \item \textbf{Versionado}: Git + GitHub
    \item \textbf{Monitoreo}: Sentry (plan gratuito 5K eventos/mes)
\end{itemize}

\textit{Justificación}: Stack completamente gratuito hasta escala significativa (miles de usuarios). Costos predecibles y escalables.

\section{Estructura de las API}

Las APIs se implementan como Edge Functions en Supabase y API Routes en Next.js:

\subsection{Endpoints de Movimientos}

\begin{table}[h!]
\centering
\small
\begin{tabular}{@{}lll@{}}
\toprule
\textbf{Método} & \textbf{Endpoint} & \textbf{Descripción} \\
\midrule
POST & /api/transactions & Crear movimiento manual \\
GET & /api/transactions & Listar movimientos (filtros: fecha, tipo, categoría) \\
GET & /api/transactions/:id & Obtener detalle de movimiento \\
PUT & /api/transactions/:id & Actualizar movimiento \\
DELETE & /api/transactions/:id & Eliminar movimiento \\
POST & /api/transactions/ocr & Procesar imagen de recibo \\
POST & /api/transactions/nlp & Procesar texto natural \\
\bottomrule
\end{tabular}
\caption{API de Transacciones}
\end{table}

\textbf{Ejemplo POST /api/transactions}:
\begin{verbatim}
Request:
{
  "type": "expense",
  "amount": 45.50,
  "currency": "USD",
  "date": "2026-01-22",
  "category_id": "uuid-categoria",
  "description": "Almuerzo restaurante",
  "payment_method": "tarjeta_credito",
  "receipt_image_url": "opcional"
}

Response 201:
{
  "id": "uuid-transaccion",
  "user_id": "uuid-usuario",
  "created_at": "2026-01-22T14:30:00Z",
  ...
}
\end{verbatim}

\subsection{Endpoints de Email Sync}

\begin{table}[h!]
\centering
\small
\begin{tabular}{@{}lll@{}}
\toprule
\textbf{Método} & \textbf{Endpoint} & \textbf{Descripción} \\
\midrule
POST & /api/email/connect & Conectar cuenta Gmail (OAuth) \\
GET & /api/email/status & Estado de integración \\
POST & /api/email/scan & Escaneo manual inmediato \\
GET & /api/email/pending & Listar movimientos pendientes aprobación \\
POST & /api/email/approve/:id & Aprobar movimiento detectado \\
POST & /api/email/reject/:id & Rechazar movimiento detectado \\
POST & /api/email/approve-bulk & Aprobar múltiples movimientos \\
\bottomrule
\end{tabular}
\caption{API de Sincronización Email}
\end{table}

\subsection{Endpoints de Presupuestos}

\begin{table}[h!]
\centering
\small
\begin{tabular}{@{}lll@{}}
\toprule
\textbf{Método} & \textbf{Endpoint} & \textbf{Descripción} \\
\midrule
POST & /api/budgets & Crear presupuesto \\
GET & /api/budgets & Listar presupuestos activos \\
GET & /api/budgets/:id & Detalle y progreso de presupuesto \\
PUT & /api/budgets/:id & Actualizar presupuesto \\
DELETE & /api/budgets/:id & Eliminar presupuesto \\
GET & /api/budgets/:id/progress & Progreso actual vs presupuesto \\
\bottomrule
\end{tabular}
\caption{API de Presupuestos}
\end{table}

\subsection{Endpoints de Insights}

\begin{table}[h!]
\centering
\small
\begin{tabular}{@{}lll@{}}
\toprule
\textbf{Método} & \textbf{Endpoint} & \textbf{Descripción} \\
\midrule
GET & /api/insights/dashboard & Dashboard mensual resumido \\
GET & /api/insights/cashflow & Flujo de caja por período \\
GET & /api/insights/ratios & Ratios financieros calculados \\
POST & /api/insights/recommendations & Generar recomendaciones IA \\
GET & /api/insights/predictions & Predicciones de gasto futuro \\
POST & /api/insights/export-pdf & Exportar reporte PDF \\
\bottomrule
\end{tabular}
\caption{API de Insights}
\end{table}

\textbf{Ejemplo GET /api/insights/ratios}:
\begin{verbatim}
Response 200:
{
  "period": "2026-01",
  "savings_rate": 0.23,
  "debt_to_asset_ratio": 0.15,
  "monthly_cashflow": 1250.00,
  "expense_to_income_ratio": 0.77,
  "calculations_date": "2026-01-22T10:00:00Z"
}
\end{verbatim}

\subsection{Endpoints de IA}

\begin{table}[h!]
\centering
\small
\begin{tabular}{@{}lll@{}}
\toprule
\textbf{Método} & \textbf{Endpoint} & \textbf{Descripción} \\
\midrule
POST & /api/ai/parse-receipt & OCR + extracción estructurada \\
POST & /api/ai/parse-text & NLP texto a transacción \\
POST & /api/ai/parse-email & Extracción info de email bancario \\
POST & /api/ai/categorize & Sugerencia de categoría \\
\bottomrule
\end{tabular}
\caption{API de Procesamiento IA}
\end{table}

\textbf{Ejemplo POST /api/ai/parse-receipt}:
\begin{verbatim}
Request:
{
  "image_base64": "data:image/jpeg;base64,/9j/4AAQ..."
}

Response 200:
{
  "merchant": "Starbucks",
  "amount": 12.50,
  "currency": "USD",
  "date": "2026-01-22",
  "category_suggestion": "alimentacion",
  "confidence": 0.92,
  "items": ["Cafe Latte", "Croissant"]
}
\end{verbatim}

\section{Consideraciones y Manejo de Errores}

\subsection{Requisitos Ambiguos}

\begin{itemize}
    \item \textbf{Frecuencia de escaneo email}: Se implementa configuración flexible (cada hora, diaria, semanal) con default diario a las 8:00 AM hora local del usuario.
    \item \textbf{Aprobación de movimientos}: Se requiere aprobación explícita uno por uno o masiva con revisión previa en UI. No hay auto-aprobación automática.
    \item \textbf{Múltiples cuentas bancarias}: Se permite registro de múltiples activos tipo ``cuenta bancaria'' para distintos bancos/cuentas.
    \item \textbf{Divisas múltiples}: Se soporta configuración de moneda predeterminada con conversión manual si el usuario registra movimientos en otras divisas.
\end{itemize}

\subsection{Manejo de Errores Técnicos}

\begin{itemize}
    \item \textbf{Fallo en OCR}: Si el procesamiento de recibo falla, se permite registro manual con opción de reintentar OCR. Se muestra mensaje claro: ``No se pudo leer el recibo automáticamente. Por favor ingresa los datos manualmente''.
    \item \textbf{Fallo en Gmail API}: Si el escaneo de email falla (token expirado, API no disponible), se notifica al usuario para re-autorizar la conexión. Se almacenan logs para debugging.
    \item \textbf{Límites de API IA}: Se implementa queue system para requests de IA. Si se alcanza límite diario de Gemini, se encolan requests para procesamiento al día siguiente con notificación al usuario.
    \item \textbf{Sincronización offline}: Movimientos creados offline se marcan como ``pending\_sync'' y se sincronizan automáticamente cuando se recupera conexión. Se muestran indicadores visuales de estado de sync.
    \item \textbf{Errores de validación}: Se validan datos en frontend y backend. Mensajes de error específicos por campo (ej: ``El monto debe ser mayor a 0'', ``Selecciona una categoría válida'').
\end{itemize}

\subsection{Seguridad y Privacidad}

\begin{itemize}
    \item \textbf{Autenticación}: Todas las APIs requieren JWT token válido. Row Level Security (RLS) en Supabase garantiza que usuarios solo accedan a sus propios datos.
    \item \textbf{Datos sensibles}: Tokens de Gmail OAuth se almacenan encriptados. Imágenes de recibos se guardan en storage privado con acceso solo al propietario.
    \item \textbf{GDPR}: Se implementa endpoint de exportación completa de datos del usuario y eliminación total de cuenta bajo demanda.
    \item \textbf{Rate limiting}: Se implementa rate limiting en Edge Functions (100 requests/minuto por usuario) para prevenir abuso.
\end{itemize}

\subsection{Escalabilidad}

\begin{itemize}
    \item \textbf{Límites tier gratuito}: Con tier gratuito de Supabase y Gemini, la app soporta hasta 200-300 usuarios activos. Más allá, se requiere upgrade a Supabase Pro (\$25/mes) y pago por uso en Gemini.
    \item \textbf{Optimización de costos IA}: Se cachean sugerencias de categorías comunes. Se agrupan múltiples requests de IA cuando sea posible.
    \item \textbf{Database indexing}: Se crean índices en campos frecuentemente consultados (user\_id, date, category\_id) para mantener queries rápidas.
\end{itemize}

\subsection{Testing y Calidad}

\begin{itemize}
    \item \textbf{Testing}: Unit tests con Jest para lógica de negocio. Integration tests para APIs críticas. E2E tests con Playwright para flujos principales.
    \item \textbf{Monitoreo}: Sentry para tracking de errores en producción. Logs estructurados en Edge Functions para debugging.
    \item \textbf{Versionado API}: Se implementa versionado en headers (API-Version: 1.0) para permitir cambios futuros sin romper clientes existentes.
\end{itemize}

\section{Estructura de Base de Datos}

La base de datos PostgreSQL en Supabase sigue un esquema relacional normalizado optimizado para consultas analíticas:

\subsection{Tablas Principales}

\subsubsection{users}
\begin{verbatim}
CREATE TABLE users (
  id UUID PRIMARY KEY DEFAULT uuid_generate_v4(),
  email VARCHAR(255) UNIQUE NOT NULL,
  full_name VARCHAR(255),
  avatar_url TEXT,
  default_currency VARCHAR(3) DEFAULT 'USD',
  timezone VARCHAR(50) DEFAULT 'UTC',
  created_at TIMESTAMPTZ DEFAULT NOW(),
  updated_at TIMESTAMPTZ DEFAULT NOW()
);
\end{verbatim}

\subsubsection{categories}
\begin{verbatim}
CREATE TABLE categories (
  id UUID PRIMARY KEY DEFAULT uuid_generate_v4(),
  user_id UUID REFERENCES users(id) ON DELETE CASCADE,
  name VARCHAR(100) NOT NULL,
  type VARCHAR(20) NOT NULL CHECK (type IN ('income', 'expense')),
  icon VARCHAR(50),
  color VARCHAR(7),
  is_default BOOLEAN DEFAULT FALSE,
  created_at TIMESTAMPTZ DEFAULT NOW(),
  UNIQUE(user_id, name, type)
);

CREATE INDEX idx_categories_user ON categories(user_id);
\end{verbatim}

\subsubsection{transactions}
\begin{verbatim}
CREATE TABLE transactions (
  id UUID PRIMARY KEY DEFAULT uuid_generate_v4(),
  user_id UUID REFERENCES users(id) ON DELETE CASCADE,
  type VARCHAR(20) NOT NULL CHECK (type IN ('income', 'expense')),
  amount DECIMAL(15,2) NOT NULL CHECK (amount > 0),
  currency VARCHAR(3) NOT NULL DEFAULT 'USD',
  date DATE NOT NULL,
  category_id UUID REFERENCES categories(id),
  description TEXT,
  payment_method VARCHAR(50),
  receipt_image_url TEXT,
  source VARCHAR(20) DEFAULT 'manual' 
    CHECK (source IN ('manual', 'ocr', 'nlp', 'email')),
  is_recurring BOOLEAN DEFAULT FALSE,
  created_at TIMESTAMPTZ DEFAULT NOW(),
  updated_at TIMESTAMPTZ DEFAULT NOW()
);

CREATE INDEX idx_transactions_user_date ON transactions(user_id, date DESC);
CREATE INDEX idx_transactions_category ON transactions(category_id);
CREATE INDEX idx_transactions_type ON transactions(type);
\end{verbatim}

\subsubsection{assets}
\begin{verbatim}
CREATE TABLE assets (
  id UUID PRIMARY KEY DEFAULT uuid_generate_v4(),
  user_id UUID REFERENCES users(id) ON DELETE CASCADE,
  name VARCHAR(255) NOT NULL,
  type VARCHAR(50) NOT NULL 
    CHECK (type IN ('bank_account', 'investment', 'property', 'other')),
  current_value DECIMAL(15,2) NOT NULL,
  currency VARCHAR(3) DEFAULT 'USD',
  description TEXT,
  created_at TIMESTAMPTZ DEFAULT NOW(),
  updated_at TIMESTAMPTZ DEFAULT NOW()
);

CREATE INDEX idx_assets_user ON assets(user_id);
\end{verbatim}

\subsubsection{liabilities}
\begin{verbatim}
CREATE TABLE liabilities (
  id UUID PRIMARY KEY DEFAULT uuid_generate_v4(),
  user_id UUID REFERENCES users(id) ON DELETE CASCADE,
  name VARCHAR(255) NOT NULL,
  type VARCHAR(50) NOT NULL 
    CHECK (type IN ('loan', 'credit_card', 'mortgage', 'other')),
  current_balance DECIMAL(15,2) NOT NULL,
  currency VARCHAR(3) DEFAULT 'USD',
  interest_rate DECIMAL(5,2),
  due_date DATE,
  description TEXT,
  created_at TIMESTAMPTZ DEFAULT NOW(),
  updated_at TIMESTAMPTZ DEFAULT NOW()
);

CREATE INDEX idx_liabilities_user ON liabilities(user_id);
\end{verbatim}

\subsubsection{budgets}
\begin{verbatim}
CREATE TABLE budgets (
  id UUID PRIMARY KEY DEFAULT uuid_generate_v4(),
  user_id UUID REFERENCES users(id) ON DELETE CASCADE,
  name VARCHAR(255) NOT NULL,
  period_type VARCHAR(20) NOT NULL 
    CHECK (period_type IN ('daily', 'weekly', 'monthly', 'yearly')),
  amount DECIMAL(15,2) NOT NULL CHECK (amount > 0),
  currency VARCHAR(3) DEFAULT 'USD',
  category_id UUID REFERENCES categories(id) ON DELETE SET NULL,
  start_date DATE NOT NULL,
  end_date DATE,
  is_active BOOLEAN DEFAULT TRUE,
  created_at TIMESTAMPTZ DEFAULT NOW(),
  updated_at TIMESTAMPTZ DEFAULT NOW()
);

CREATE INDEX idx_budgets_user_active ON budgets(user_id, is_active);
CREATE INDEX idx_budgets_period ON budgets(period_type, start_date);
\end{verbatim}

\subsubsection{financial\_goals}
\begin{verbatim}
CREATE TABLE financial_goals (
  id UUID PRIMARY KEY DEFAULT uuid_generate_v4(),
  user_id UUID REFERENCES users(id) ON DELETE CASCADE,
  name VARCHAR(255) NOT NULL,
  target_amount DECIMAL(15,2) NOT NULL CHECK (target_amount > 0),
  current_amount DECIMAL(15,2) DEFAULT 0,
  currency VARCHAR(3) DEFAULT 'USD',
  target_date DATE,
  description TEXT,
  is_completed BOOLEAN DEFAULT FALSE,
  created_at TIMESTAMPTZ DEFAULT NOW(),
  updated_at TIMESTAMPTZ DEFAULT NOW()
);

CREATE INDEX idx_goals_user ON financial_goals(user_id, is_completed);
\end{verbatim}

\subsubsection{email\_integrations}
\begin{verbatim}
CREATE TABLE email_integrations (
  id UUID PRIMARY KEY DEFAULT uuid_generate_v4(),
  user_id UUID REFERENCES users(id) ON DELETE CASCADE,
  email_provider VARCHAR(50) DEFAULT 'gmail',
  email_address VARCHAR(255) NOT NULL,
  access_token_encrypted TEXT NOT NULL,
  refresh_token_encrypted TEXT,
  scan_schedule VARCHAR(20) DEFAULT 'daily',
  last_scan_at TIMESTAMPTZ,
  is_active BOOLEAN DEFAULT TRUE,
  created_at TIMESTAMPTZ DEFAULT NOW(),
  updated_at TIMESTAMPTZ DEFAULT NOW(),
  UNIQUE(user_id, email_provider)
);

CREATE INDEX idx_email_integrations_user ON email_integrations(user_id);
\end{verbatim}

\subsubsection{pending\_transactions}
\begin{verbatim}
CREATE TABLE pending_transactions (
  id UUID PRIMARY KEY DEFAULT uuid_generate_v4(),
  user_id UUID REFERENCES users(id) ON DELETE CASCADE,
  email_integration_id UUID REFERENCES email_integrations(id),
  type VARCHAR(20) NOT NULL,
  amount DECIMAL(15,2) NOT NULL,
  currency VARCHAR(3) DEFAULT 'USD',
  date DATE NOT NULL,
  category_id UUID REFERENCES categories(id),
  description TEXT,
  source_email_subject TEXT,
  source_email_date TIMESTAMPTZ,
  confidence_score DECIMAL(3,2),
  status VARCHAR(20) DEFAULT 'pending' 
    CHECK (status IN ('pending', 'approved', 'rejected')),
  created_at TIMESTAMPTZ DEFAULT NOW()
);

CREATE INDEX idx_pending_trans_user_status 
  ON pending_transactions(user_id, status);
\end{verbatim}

\subsubsection{ai\_recommendations}
\begin{verbatim}
CREATE TABLE ai_recommendations (
  id UUID PRIMARY KEY DEFAULT uuid_generate_v4(),
  user_id UUID REFERENCES users(id) ON DELETE CASCADE,
  recommendation_type VARCHAR(50) NOT NULL,
  title VARCHAR(255) NOT NULL,
  description TEXT NOT NULL,
  priority VARCHAR(20) DEFAULT 'medium' 
    CHECK (priority IN ('low', 'medium', 'high')),
  is_read BOOLEAN DEFAULT FALSE,
  is_dismissed BOOLEAN DEFAULT FALSE,
  generated_at TIMESTAMPTZ DEFAULT NOW(),
  expires_at TIMESTAMPTZ
);

CREATE INDEX idx_recommendations_user 
  ON ai_recommendations(user_id, is_dismissed, is_read);
\end{verbatim}

\subsection{Vistas Materializadas para Analytics}

\subsubsection{monthly\_summary}
\begin{verbatim}
CREATE MATERIALIZED VIEW monthly_summary AS
SELECT 
  user_id,
  DATE_TRUNC('month', date) as month,
  SUM(CASE WHEN type = 'income' THEN amount ELSE 0 END) 
    as total_income,
  SUM(CASE WHEN type = 'expense' THEN amount ELSE 0 END) 
    as total_expenses,
  SUM(CASE WHEN type = 'income' THEN amount ELSE -amount END) 
    as net_cashflow,
  COUNT(*) as transaction_count
FROM transactions
GROUP BY user_id, DATE_TRUNC('month', date);

CREATE UNIQUE INDEX idx_monthly_summary_user_month 
  ON monthly_summary(user_id, month);
\end{verbatim}

\subsection{Row Level Security (RLS)}

Todas las tablas implementan RLS para garantizar aislamiento de datos:

\begin{verbatim}
-- Ejemplo para transactions
ALTER TABLE transactions ENABLE ROW LEVEL SECURITY;

CREATE POLICY "Users can view own transactions"
  ON transactions FOR SELECT
  USING (auth.uid() = user_id);

CREATE POLICY "Users can insert own transactions"
  ON transactions FOR INSERT
  WITH CHECK (auth.uid() = user_id);

CREATE POLICY "Users can update own transactions"
  ON transactions FOR UPDATE
  USING (auth.uid() = user_id)
  WITH CHECK (auth.uid() = user_id);

CREATE POLICY "Users can delete own transactions"
  ON transactions FOR DELETE
  USING (auth.uid() = user_id);
\end{verbatim}

\section{Estructura de Vistas y Pantallas de la PWA}

La arquitectura de navegación sigue un patrón de tabs principales con sub-navegación jerárquica:

\subsection{Navegación Principal (Bottom Navigation)}

\begin{enumerate}
  \item \textbf{Home/Dashboard}
  \item \textbf{Transacciones}
  \item \textbf{Presupuestos}
  \item \textbf{Insights}
  \item \textbf{Perfil}
\end{enumerate}

\subsection{Flujo de Pantallas Detallado}

\subsubsection{Autenticación y Onboarding (Auth \& Onboarding Flow)}

\begin{itemize}
  \item \textbf{Splash Screen (Primera carga)}
    \begin{itemize}
      \item Logo de Fintrack centrado con animación de fade-in
      \item Loading spinner sutil debajo del logo
      \item Verificación de autenticación en background
      \item Duración: 1.5-2 segundos
      \item Transición suave a Welcome o Dashboard según estado de autenticación
    \end{itemize}
  
  \item \textbf{Welcome Screen (Primera vez)}
    \begin{itemize}
      \item Hero visual con ilustración de finanzas personales
      \item Título: "Bienvenido a Fintrack"
      \item Subtítulo: "Toma control total de tus finanzas personales"
      \item Botón principal: "Comenzar" (CTA destacado)
      \item Link inferior: "Ya tengo cuenta - Iniciar sesión"
      \item Indicadores de valor: "Seguro • Privado • Fácil de usar"
    \end{itemize}
  
  \item \textbf{Onboarding Slides (Carrusel de 3 slides)}
    \begin{itemize}
      \item \textbf{Slide 1: Registro Inteligente}
        \begin{itemize}
          \item Ilustración: Teléfono escaneando recibo
          \item Título: "Registra tus gastos en segundos"
          \item Descripción: "Escanea recibos, usa texto natural o conecta tu email. Fintrack hace el trabajo pesado por ti."
        \end{itemize}
      
      \item \textbf{Slide 2: Control Total}
        \begin{itemize}
          \item Ilustración: Dashboard con gráficos
          \item Título: "Visualiza tu salud financiera"
          \item Descripción: "Presupuestos inteligentes, seguimiento de metas y análisis detallados de tus finanzas."
        \end{itemize}
      
      \item \textbf{Slide 3: Insights con IA}
        \begin{itemize}
          \item Ilustración: Robot/IA con estadísticas
          \item Título: "Recomendaciones personalizadas"
          \item Descripción: "Recibe consejos inteligentes basados en tus hábitos para alcanzar tus objetivos financieros."
        \end{itemize}
      
      \item Navegación: Dots indicator + botón "Siguiente"
      \item Último slide: Botón "Crear mi cuenta"
      \item Todas las slides: Link "Saltar" en esquina superior derecha
    \end{itemize}
  
  \item \textbf{Registro (Sign Up)}
    \begin{itemize}
      \item Header con botón volver atrás
      \item Título: "Crea tu cuenta"
      \item Campo: Nombre completo (validación: mínimo 2 palabras)
      \item Campo: Email (validación en tiempo real)
      \item Campo: Contraseña (indicador de fortaleza visual)
      \item Campo: Confirmar contraseña
      \item Checkbox: "Acepto términos y condiciones" (obligatorio)
      \item Botón principal: "Crear cuenta"
      \item Divisor visual con texto "o continúa con"
      \item Botones OAuth:
        \begin{itemize}
          \item Google (icono + "Continuar con Google")
          \item Apple (icono + "Continuar con Apple")
        \end{itemize}
      \item Link inferior: "¿Ya tienes cuenta? Inicia sesión"
      \item Mensajes de error inline por campo
    \end{itemize}
  
  \item \textbf{Login (Sign In)}
    \begin{itemize}
      \item Header con botón volver atrás
      \item Título: "¡Bienvenido de nuevo!"
      \item Campo: Email
      \item Campo: Contraseña con toggle mostrar/ocultar
      \item Checkbox: "Recordarme"
      \item Link: "¿Olvidaste tu contraseña?" (alineado a la derecha)
      \item Botón principal: "Iniciar sesión"
      \item Divisor visual con texto "o continúa con"
      \item Botones OAuth (Google, Apple)
      \item Link inferior: "¿No tienes cuenta? Regístrate"
      \item Loading state en botón durante autenticación
      \item Error toast si credenciales incorrectas
    \end{itemize}
  
  \item \textbf{Recuperación de Contraseña (Password Recovery)}
    \begin{itemize}
      \item Header con botón volver atrás
      \item Título: "Recupera tu contraseña"
      \item Ilustración: Sobre de email
      \item Descripción: "Ingresa tu email y te enviaremos instrucciones para restablecer tu contraseña"
      \item Campo: Email
      \item Botón: "Enviar instrucciones"
      \item Success screen tras envío:
        \begin{itemize}
          \item Checkmark animado
          \item Mensaje: "Email enviado"
          \item Descripción: "Revisa tu bandeja de entrada y sigue las instrucciones"
          \item Botón: "Volver al inicio de sesión"
          \item Link: "¿No recibiste el email? Reenviar"
        \end{itemize}
    \end{itemize}
  
  \item \textbf{Onboarding Setup (Configuración Inicial - Post Registro)}
    \begin{itemize}
      \item \textbf{Paso 1/4: Bienvenida Personalizada}
        \begin{itemize}
          \item Título: "¡Hola [Nombre]!"
          \item Subtítulo: "Vamos a configurar tu perfil financiero"
          \item Progress bar: 25\% completado
          \item Botón: "Continuar"
        \end{itemize}
      
      \item \textbf{Paso 2/4: Configuración de Moneda}
        \begin{itemize}
          \item Título: "¿Cuál es tu moneda principal?"
          \item Dropdown/Selector con búsqueda de monedas
          \item Monedas destacadas: USD, EUR, PEN, MXN, etc.
          \item Preview: "Ejemplo: \$1,234.56 USD"
          \item Progress bar: 50\%
          \item Botones: "Atrás" y "Siguiente"
        \end{itemize}
      
      \item \textbf{Paso 3/4: Categorías Favoritas}
        \begin{itemize}
          \item Título: "Selecciona tus categorías de gasto más comunes"
          \item Subtítulo: "Puedes agregar más después"
          \item Grid de categorías con iconos y nombres:
            \begin{itemize}
              \item Alimentación, Transporte, Vivienda
              \item Entretenimiento, Salud, Educación
              \item Servicios, Compras, Otros
            \end{itemize}
          \item Interacción: Tap para seleccionar (efecto visual)
          \item Mínimo 3 categorías seleccionadas para continuar
          \item Progress bar: 75\%
          \item Botones: "Atrás" y "Siguiente"
        \end{itemize}
      
      \item \textbf{Paso 4/4: Configuración de Notificaciones}
        \begin{itemize}
          \item Título: "¿Quieres recibir notificaciones?"
          \item Lista de toggles:
            \begin{itemize}
              \item "Alertas de presupuesto" (recomendado)
              \item "Recordatorios de metas"
              \item "Resumen semanal"
              \item "Recomendaciones de IA"
            \end{itemize}
          \item Nota: "Puedes cambiar esto después en configuración"
          \item Progress bar: 100\%
          \item Botón principal: "Comenzar a usar Fintrack"
        \end{itemize}
      
      \item \textbf{Success Screen (Configuración Completa)}
        \begin{itemize}
          \item Animación celebratoria (confetti o checkmark)
          \item Título: "¡Todo listo!"
          \item Mensaje: "Tu perfil financiero está configurado"
          \item Botón: "Ir al Dashboard"
          \item Auto-redirect tras 2 segundos
        \end{itemize}
    \end{itemize}
  
  \item \textbf{Verificación de Email (Opcional)}
    \begin{itemize}
      \item Banner superior en Dashboard si email no verificado
      \item Mensaje: "Verifica tu email para desbloquear todas las funciones"
      \item Botón: "Enviar email de verificación"
      \item Puede ser omitido temporalmente con "X"
    \end{itemize}
\end{itemize}

\subsubsection{Tab 1: Home/Dashboard}

\begin{itemize}
  \item \textbf{Dashboard Principal}
    \begin{itemize}
      \item Header con saludo personalizado + foto perfil
      \item Card de resumen del mes actual (ingresos, egresos, balance)
      \item Gráfico de flujo de caja últimos 6 meses (line chart)
      \item Lista de transacciones recientes (últimas 5)
      \item FAB (Floating Action Button) para agregar transacción rápida
      \item Widget de patrimonio neto actual
      \item Notificaciones/alertas pendientes (badge)
    \end{itemize}
\end{itemize}

\subsubsection{Tab 2: Transacciones}

\begin{itemize}
  \item \textbf{Lista de Transacciones}
    \begin{itemize}
      \item Filtros superiores (fecha, tipo, categoría, monto)
      \item Lista agrupada por fecha (hoy, ayer, esta semana, etc.)
      \item Cada item muestra: icono categoría, descripción, monto, fecha
      \item Swipe actions: editar, eliminar
      \item FAB con menú desplegable: agregar manual, escanear recibo, texto natural
      \item Búsqueda por descripción
    \end{itemize}
  
  \item \textbf{Agregar Transacción Manual}
    \begin{itemize}
      \item Toggle ingreso/egreso
      \item Input de monto (numpad grande)
      \item Selector de fecha (date picker)
      \item Selector de categoría (grid con iconos)
      \item Campo descripción (opcional)
      \item Selector método de pago
      \item Botón adjuntar imagen
      \item Botón guardar
    \end{itemize}
  
  \item \textbf{Escanear Recibo (OCR)}
    \begin{itemize}
      \item Vista de cámara con overlay guía
      \item Captura de foto
      \item Preview con loading mientras procesa
      \item Formulario pre-llenado con datos extraídos
      \item Indicador de confianza de OCR
      \item Botón confirmar/editar
    \end{itemize}
  
  \item \textbf{Entrada de Texto Natural}
    \begin{itemize}
      \item Text area grande con placeholder ejemplos
      \item Botón procesar con loading
      \item Confirmación de datos interpretados
      \item Sugerencia de categoría por IA
    \end{itemize}
  
  \item \textbf{Detalle de Transacción}
    \begin{itemize}
      \item Todos los campos en modo lectura
      \item Imagen del recibo (si existe) con zoom
      \item Botón editar
      \item Botón eliminar con confirmación
      \item Metadata: fecha creación, fuente (manual/OCR/email)
    \end{itemize}
\end{itemize}

\subsubsection{Tab 3: Presupuestos}

\begin{itemize}
  \item \textbf{Vista Principal Presupuestos}
    \begin{itemize}
      \item Tabs secundarios: Diario, Semanal, Mensual, Anual
      \item Cards de presupuestos activos con progress bar circular
      \item Indicador visual: verde (<80\%), amarillo (80-100\%), rojo (>100\%)
      \item FAB para crear nuevo presupuesto
      \item Opción ver presupuestos vencidos/inactivos
    \end{itemize}
  
  \item \textbf{Crear/Editar Presupuesto}
    \begin{itemize}
      \item Nombre del presupuesto
      \item Selector de período (diario/semanal/mensual/anual)
      \item Input de monto
      \item Selector de categoría (opcional, si no selecciona = presupuesto total)
      \item Date picker fecha inicio
      \item Toggle repetir automáticamente
      \item Configuración de alertas (80\%, 100\%)
    \end{itemize}
  
  \item \textbf{Detalle de Presupuesto}
    \begin{itemize}
      \item Progress bar grande con porcentaje
      \item Monto gastado vs monto total
      \item Proyección: "A este ritmo terminarás el presupuesto en X días"
      \item Lista de transacciones asociadas a este presupuesto
      \item Gráfico de gasto diario vs promedio permitido
      \item Botones: editar, eliminar, pausar
    \end{itemize}
\end{itemize}

\subsubsection{Tab 4: Insights}

\begin{itemize}
  \item \textbf{Dashboard de Insights}
    \begin{itemize}
      \item Selector de período (mes, trimestre, año, personalizado)
      \item Card de ratios financieros principales:
        \begin{itemize}
          \item Tasa de ahorro
          \item Ratio de endeudamiento
          \item Flujo de caja mensual
        \end{itemize}
      \item Gráfico pie chart de distribución de gastos por categoría
      \item Gráfico de tendencia de gastos (últimos 6 meses)
      \item Top 5 categorías de mayor gasto
      \item Comparación mes actual vs mes anterior
    \end{itemize}
  
  \item \textbf{Recomendaciones IA}
    \begin{itemize}
      \item Lista de cards con recomendaciones priorizadas
      \item Cada card muestra: título, descripción, prioridad (color), icono
      \item Actions: marcar como leída, descartar, ver detalle
      \item Filtros: no leídas, prioridad alta
    \end{itemize}
  
  \item \textbf{Predicciones}
    \begin{itemize}
      \item Gráfico con proyección de gastos próximos 3 meses
      \item Alertas predictivas: "Podrías exceder tu presupuesto de X"
      \item Predicción de alcance de metas financieras
    \end{itemize}
  
  \item \textbf{Exportar Reportes}
    \begin{itemize}
      \item Selector de período
      \item Checkbox de secciones a incluir (transacciones, presupuestos, etc.)
      \item Botón generar PDF con loading
      \item Opción compartir o descargar
    \end{itemize}
  
  \item \textbf{Activos y Pasivos}
    \begin{itemize}
      \item Card de patrimonio neto total destacado
      \item Lista de activos con valores actuales
      \item Lista de pasivos con saldos pendientes
      \item Gráfico de evolución de patrimonio neto (últimos 12 meses)
      \item FAB para agregar activo/pasivo
    \end{itemize}
  
  \item \textbf{Metas Financieras}
    \begin{itemize}
      \item Lista de metas activas con progress bar
      \item Cada meta muestra: nombre, monto actual/objetivo, fecha límite, \% progreso
      \item Indicador visual de proyección (alcanzable, en riesgo, excedido)
      \item FAB crear nueva meta
    \end{itemize}
  
  \item \textbf{Detalle de Meta}
    \begin{itemize}
      \item Progress circular grande
      \item Desglose: monto ahorrado, faltante, días restantes
      \item Cálculo: "Necesitas ahorrar \$X por mes para cumplir la meta"
      \item Timeline de aportaciones realizadas
      \item Botón agregar aportación manual
      \item Gráfico de proyección vs real
    \end{itemize}
\end{itemize}

\subsubsection{Tab 5: Perfil}

\begin{itemize}
  \item \textbf{Vista de Perfil}
    \begin{itemize}
      \item Header con foto, nombre, email
      \item Secciones con iconos:
        \begin{itemize}
          \item Configuración de cuenta
          \item Categorías personalizadas
          \item Sincronización de email
          \item Preferencias de la app
          \item Seguridad y privacidad
          \item Ayuda y soporte
          \item Acerca de Fintrack
          \item Cerrar sesión
        \end{itemize}
    \end{itemize}
  
  \item \textbf{Configuración de Cuenta}
    \begin{itemize}
      \item Editar nombre completo
      \item Cambiar foto de perfil
      \item Cambiar contraseña
      \item Configurar moneda predeterminada
      \item Configurar zona horaria
    \end{itemize}
  
  \item \textbf{Categorías Personalizadas}
    \begin{itemize}
      \item Lista de categorías (predeterminadas en gris, personalizadas editables)
      \item Cada item: icono, nombre, tipo (ingreso/egreso), color
      \item FAB para crear nueva categoría
      \item Swipe para editar/eliminar (solo personalizadas)
    \end{itemize}
  
  \item \textbf{Sincronización de Email}
    \begin{itemize}
      \item Status de conexión (conectado/desconectado)
      \item Email conectado actual
      \item Configuración de frecuencia de escaneo
      \item Última fecha de escaneo
      \item Botón conectar/desconectar Gmail
      \item Botón escanear ahora (manual)
      \item Lista de movimientos pendientes de aprobación con badge
    \end{itemize}
  
  \item \textbf{Movimientos Pendientes Email}
    \begin{itemize}
      \item Lista de transacciones detectadas en emails
      \item Cada item: fecha, monto, descripción, categoría sugerida, confianza
      \item Actions por item: aprobar (check), rechazar (X), editar
      \item Botón aprobar seleccionados (múltiple)
      \item Filtros: solo alta confianza, por tipo
    \end{itemize}
  
  \item \textbf{Preferencias de la App}
    \begin{itemize}
      \item Toggle modo oscuro/claro
      \item Configuración de notificaciones push
      \item Idioma de la app
      \item Formato de fecha y moneda
    \end{itemize}
  
  \item \textbf{Seguridad y Privacidad}
    \begin{itemize}
      \item Habilitar autenticación biométrica
      \item Gestión de sesiones activas
      \item Exportar todos mis datos (GDPR)
      \item Eliminar mi cuenta (con confirmación doble)
    \end{itemize}
\end{itemize}

\subsection{Componentes Transversales}

\begin{itemize}
  \item \textbf{Modal de Agregar Rápido (FAB)}
    \begin{itemize}
      \item Botón flotante persistente en esquina inferior derecha
      \item Al presionar despliega menú radial con opciones:
        \begin{itemize}
          \item Ingreso rápido
          \item Egreso rápido
          \item Escanear recibo
          \item Texto natural
        \end{itemize}
    \end{itemize}
  
  \item \textbf{Notificaciones In-App}
    \begin{itemize}
      \item Toast notifications para confirmaciones
      \item Banner persistente para alertas de presupuesto
      \item Modal para aprobar movimientos de email detectados
    \end{itemize}
  
  \item \textbf{Bottom Sheet}
    \begin{itemize}
      \item Usado para filtros avanzados
      \item Selección de categorías
      \item Confirmaciones de eliminación
    \end{itemize}
  
  \item \textbf{Skeleton Loaders}
    \begin{itemize}
      \item Placeholders animados durante carga de datos
      \item Mejora percepción de velocidad
    \end{itemize}
\end{itemize}

\subsection{Diseño Responsive}

\begin{itemize}
  \item \textbf{Mobile (< 640px)}: Navigation bottom tabs, stack layout, swipe gestures
  \item \textbf{Tablet (640-1024px)}: Sidebar navigation, grid de 2 columnas para cards
  \item \textbf{Desktop (> 1024px)}: Sidebar persistente, dashboard en grid 3 columnas, charts más grandes
\end{itemize}

\subsection{Modo Offline}

\begin{itemize}
  \item Banner superior indica "Sin conexión - Los cambios se sincronizarán"
  \item Transacciones creadas offline se marcan con icono de sincronización pendiente
  \item Al recuperar conexión, sincronización automática con indicador de progreso
  \item Caché de últimos 30 días de transacciones para visualización offline
\end{itemize}

\end{document}