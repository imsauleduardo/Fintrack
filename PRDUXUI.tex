\documentclass[11pt,a4paper]{article}
\usepackage[utf8]{inputenc}
\usepackage[spanish]{babel}
\usepackage{geometry}
\usepackage{enumitem}
\usepackage{booktabs}
\usepackage{hyperref}

\geometry{margin=2.5cm}

\title{\textbf{Product Requirements Document (PRD) UX/UI}\\
\large Fintrack - Aplicación de Control de Finanzas Personales}
\author{Saul Eduardo Fernandez}
\date{\today}

\begin{document}

\maketitle

\section{Guía de UX/UI y Diseño Visual}

Esta sección define los principios de diseño, sistema visual y directrices de experiencia de usuario para Fintrack, con un enfoque minimalista que prioriza claridad, funcionalidad y accesibilidad.

\subsection{Principios de Diseño}

\subsubsection{Minimalismo Funcional}

\begin{itemize}
  \item \textbf{Simplicidad sobre complejidad}: Cada elemento en pantalla debe tener un propósito claro. Eliminar decoraciones innecesarias.
  \item \textbf{Jerarquía visual clara}: Usar tamaño, peso y espaciado para guiar la atención, no múltiples colores.
  \item \textbf{Espacio en blanco generoso}: Mínimo 16px entre elementos relacionados, 32px entre secciones.
  \item \textbf{Contenido primero}: El diseño sirve al contenido, no al revés. Los datos financieros son la estrella.
  \item \textbf{Progressive disclosure}: Mostrar información en capas, de lo general a lo específico.
\end{itemize}

\subsubsection{Claridad y Legibilidad}

\begin{itemize}
  \item \textbf{Tipografía limpia}: Fuentes sans-serif de alta legibilidad en todos los tamaños.
  \item \textbf{Contraste adecuado}: Mínimo WCAG AA (4.5:1 para texto normal, 3:1 para texto grande).
  \item \textbf{Información escaneable}: Usuarios deben poder entender estado financiero en 3 segundos.
  \item \textbf{Números destacados}: Los montos financieros deben ser lo más legible en cada pantalla.
\end{itemize}

\subsubsection{Confianza y Seguridad}

\begin{itemize}
  \item \textbf{Estética profesional}: El diseño debe transmitir seriedad y confiabilidad.
  \item \textbf{Feedback constante}: Cada acción del usuario recibe respuesta visual inmediata.
  \item \textbf{Confirmaciones en acciones críticas}: Delete, aprobación de gastos grandes, etc.
  \item \textbf{Transparencia}: Mostrar claramente fuente de datos (manual, OCR, email, IA).
\end{itemize}

\subsection{Sistema de Diseño}

\subsubsection{Paleta de Colores}

\textbf{Esquema Minimalista Monocromático + Acentos Funcionales}

\begin{itemize}
  \item \textbf{Modo Claro (Default)}
    \begin{itemize}
      \item Background principal: \texttt{\#FFFFFF} (blanco puro)
      \item Background secundario: \texttt{\#F8F9FA} (gris ultra claro)
      \item Texto primario: \texttt{\#1A1A1A} (negro suave, no puro)
      \item Texto secundario: \texttt{\#6B7280} (gris medio)
      \item Texto terciario: \texttt{\#9CA3AF} (gris claro)
      \item Bordes/divisores: \texttt{\#E5E7EB} (gris muy claro)
    \end{itemize}
  
  \item \textbf{Modo Oscuro}
    \begin{itemize}
      \item Background principal: \texttt{\#0A0A0A} (negro profundo)
      \item Background secundario: \texttt{\#1A1A1A} (gris muy oscuro)
      \item Texto primario: \texttt{\#F8F9FA} (blanco suave)
      \item Texto secundario: \texttt{\#9CA3AF} (gris medio)
      \item Texto terciario: \texttt{\#6B7280} (gris oscuro)
      \item Bordes/divisores: \texttt{\#2D2D2D} (gris oscuro sutil)
    \end{itemize}
  
  \item \textbf{Colores Funcionales (Mínimos y estratégicos)}
    \begin{itemize}
      \item \textbf{Primary/Brand}: \texttt{\#3B82F6} (azul confiable, corporativo)
        \begin{itemize}
          \item Uso: CTAs principales, links, elementos interactivos
          \item Hover: \texttt{\#2563EB}
          \item Disabled: \texttt{\#93C5FD}
        \end{itemize}
      
      \item \textbf{Success/Ingresos}: \texttt{\#10B981} (verde éxito)
        \begin{itemize}
          \item Uso: Ingresos, confirmaciones, metas alcanzadas
          \item Background suave: \texttt{\#D1FAE5}
        \end{itemize}
      
      \item \textbf{Danger/Egresos}: \texttt{\#EF4444} (rojo advertencia)
        \begin{itemize}
          \item Uso: Egresos, alertas, eliminaciones
          \item Background suave: \texttt{\#FEE2E2}
        \end{itemize}
      
      \item \textbf{Warning}: \texttt{\#F59E0B} (amarillo alerta)
        \begin{itemize}
          \item Uso: Advertencias, presupuestos al 80\%
          \item Background suave: \texttt{\#FEF3C7}
        \end{itemize}
      
      \item \textbf{Info}: \texttt{\#6366F1} (índigo informativo)
        \begin{itemize}
          \item Uso: Tips, información contextual, insights IA
          \item Background suave: \texttt{\#E0E7FF}
        \end{itemize}
    \end{itemize}
  
  \item \textbf{Regla de uso de color}:
    \begin{itemize}
      \item 90\% de la interfaz en escala de grises (negro, blanco, grises)
      \item 10\% colores funcionales, solo cuando añaden significado
      \item Máximo 2 colores por pantalla (excluyendo grises)
    \end{itemize}
\end{itemize}

\subsubsection{Tipografía}

\begin{itemize}
  \item \textbf{Fuente principal}: \texttt{Inter} (Google Fonts - gratis, excelente legibilidad)
    \begin{itemize}
      \item Alternativa: \texttt{SF Pro} (iOS), \texttt{Roboto} (Android) - nativas del sistema
    \end{itemize}
  
  \item \textbf{Escala tipográfica} (escala modular 1.25):
    \begin{itemize}
      \item Display (Headers grandes): 32px / 2rem, weight 700, line-height 1.2
      \item H1: 24px / 1.5rem, weight 600, line-height 1.3
      \item H2: 20px / 1.25rem, weight 600, line-height 1.4
      \item H3: 18px / 1.125rem, weight 600, line-height 1.4
      \item Body Large (montos destacados): 18px / 1.125rem, weight 500, line-height 1.5
      \item Body (texto principal): 16px / 1rem, weight 400, line-height 1.6
      \item Body Small: 14px / 0.875rem, weight 400, line-height 1.5
      \item Caption: 12px / 0.75rem, weight 400, line-height 1.4
    \end{itemize}
  
  \item \textbf{Números y montos}: 
    \begin{itemize}
      \item Usar \texttt{font-variant-numeric: tabular-nums} para alineación
      \item Weight 600 para montos destacados
      \item Siempre mostrar 2 decimales para consistencia: \$1,234.56
    \end{itemize}
  
  \item \textbf{Reglas de uso}:
    \begin{itemize}
      \item Máximo 3 pesos (weights) por pantalla: regular (400), medium (500), semibold (600)
      \item Line-height generoso para lectura cómoda
      \item Letter-spacing normal (no ajustes salvo títulos grandes: -0.02em)
    \end{itemize}
\end{itemize}

\subsubsection{Espaciado y Grid}

\begin{itemize}
  \item \textbf{Sistema de espaciado base 4px} (múltiplos de 4):
    \begin{itemize}
      \item 4px: Espaciado mínimo entre elementos relacionados
      \item 8px: Padding interno de componentes pequeños
      \item 12px: Espaciado entre elementos en grupos
      \item 16px: Padding estándar, margen entre secciones relacionadas
      \item 24px: Separación entre grupos de contenido
      \item 32px: Separación entre secciones principales
      \item 48px: Espaciado entre bloques importantes
      \item 64px: Márgenes grandes, separadores visuales
    \end{itemize}
  
  \item \textbf{Grid responsive}:
    \begin{itemize}
      \item Mobile: 16px padding lateral, 1 columna
      \item Tablet: 24px padding lateral, 2 columnas con 24px gap
      \item Desktop: Container max-width 1280px, 3 columnas con 32px gap
    \end{itemize}
  
  \item \textbf{Altura de elementos táctiles}:
    \begin{itemize}
      \item Mínimo 44px (iOS guideline) / 48px (Material Design)
      \item Botones principales: 48px altura
      \item Botones secundarios: 40px altura
      \item Input fields: 48px altura para fácil interacción
    \end{itemize}
\end{itemize}

\subsubsection{Componentes UI}

\begin{itemize}
  \item \textbf{Botones}
    \begin{itemize}
      \item \textit{Primary}: Background color brand, texto blanco, border-radius 8px, padding 12px 24px
      \item \textit{Secondary}: Background transparent, borde color brand, texto color brand
      \item \textit{Ghost}: Sin background ni borde, solo texto color brand
      \item \textit{Destructive}: Background color danger para acciones peligrosas
      \item Estados: Default, Hover (opacidad 90\%), Active, Disabled (opacidad 40\%), Loading (spinner)
      \item Sombra sutil en hover: \texttt{box-shadow: 0 2px 8px rgba(0,0,0,0.08)}
    \end{itemize}
  
  \item \textbf{Cards}
    \begin{itemize}
      \item Background: blanco (modo claro) / gris oscuro (modo oscuro)
      \item Border: \texttt{1px solid} color borde
      \item Border-radius: 12px (suave, moderno)
      \item Padding interno: 16px (mobile) / 20px (desktop)
      \item Sombra mínima: \texttt{box-shadow: 0 1px 3px rgba(0,0,0,0.04)}
      \item Hover elevation (opcional): \texttt{box-shadow: 0 4px 12px rgba(0,0,0,0.08)}
    \end{itemize}
  
  \item \textbf{Input Fields}
    \begin{itemize}
      \item Height: 48px
      \item Border: \texttt{1px solid} color borde
      \item Border-radius: 8px
      \item Padding: 12px 16px
      \item Focus state: borde color brand, outline offset 2px
      \item Error state: borde color danger, mensaje de error debajo en color danger
      \item Label: 14px, weight 500, margin-bottom 8px
      \item Placeholder: texto terciario
    \end{itemize}
  
  \item \textbf{Icons}
    \begin{itemize}
      \item Librería: Lucide Icons o Heroicons (consistencia de estilo)
      \item Tamaños: 16px (small), 20px (medium), 24px (large), 32px (xlarge)
      \item Stroke-width: 2px (consistente con estilo minimalista)
      \item Color: hereda del texto padre, o color funcional cuando necesario
    \end{itemize}
  
  \item \textbf{Progress Bars}
    \begin{itemize}
      \item Altura: 8px (slim) o 12px (destacado)
      \item Background: gris claro
      \item Fill: gradiente sutil del color funcional o color sólido
      \item Border-radius: 999px (píldora completa)
      \item Animación suave en cambio de progreso (transition 0.3s ease)
    \end{itemize}
  
  \item \textbf{Badges/Tags}
    \begin{itemize}
      \item Pequeños, discretos: padding 4px 8px
      \item Border-radius: 4px
      \item Font-size: 12px, weight 500
      \item Background: color funcional suave (ej: \texttt{\#D1FAE5} para ingresos)
      \item Texto: color funcional oscuro para contraste
    \end{itemize}
  
  \item \textbf{Modals/Dialogs}
    \begin{itemize}
      \item Overlay: \texttt{rgba(0,0,0,0.5)} con backdrop-blur 4px
      \item Container: background blanco, border-radius 16px, max-width 480px
      \item Padding: 24px
      \item Sombra profunda: \texttt{box-shadow: 0 20px 60px rgba(0,0,0,0.3)}
      \item Animación entrada: fade + scale desde 0.95 a 1
    \end{itemize}
  
  \item \textbf{Bottom Sheets (Mobile)}
    \begin{itemize}
      \item Aparecen desde abajo con animación slide-up
      \item Handle visual arriba: barra gris de 32px x 4px
      \item Border-radius superior: 16px
      \item Swipe-down para cerrar
    \end{itemize}
\end{itemize}

\subsection{Patrones de Interacción}

\subsubsection{Navegación}

\begin{itemize}
  \item \textbf{Mobile}: Bottom navigation con 5 tabs, iconos + labels
  \item \textbf{Tablet/Desktop}: Sidebar colapsable en left side
  \item \textbf{Gestos}:
    \begin{itemize}
      \item Swipe horizontal entre tabs (móvil)
      \item Swipe en items de lista: revelar acciones (editar, eliminar)
      \item Pull-to-refresh en listas
      \item Long-press para menú contextual
    \end{itemize}
  \item \textbf{Transiciones}:
    \begin{itemize}
      \item Navegación entre pantallas: slide horizontal (150ms ease-out)
      \item Modals: fade + scale (200ms ease-out)
      \item Collapses/expands: height transition (250ms ease-in-out)
    \end{itemize}
\end{itemize}

\subsubsection{Feedback Visual}

\begin{itemize}
  \item \textbf{Loading states}:
    \begin{itemize}
      \item Skeleton loaders para contenido (no spinners genéricos)
      \item Progress bar linear en top de pantalla para operaciones largas
      \item Spinner solo en botones durante submit
    \end{itemize}
  
  \item \textbf{Confirmación de acciones}:
    \begin{itemize}
      \item Toast notifications: 3 segundos, bottom-center (mobile), top-right (desktop)
      \item Success: checkmark verde + mensaje breve
      \item Error: icono X rojo + mensaje explicativo
      \item Info: icono i azul + contexto adicional
    \end{itemize}
  
  \item \textbf{Estados vacíos (Empty states)}:
    \begin{itemize}
      \item Ilustración simple y minimalista
      \item Texto explicativo: "Aún no tienes transacciones"
      \item CTA claro: "Agregar mi primera transacción"
      \item Centrado vertical y horizontalmente
    \end{itemize}
  
  \item \textbf{Micro-interacciones}:
    \begin{itemize}
      \item Botones: scale 0.98 en active state
      \item Checkboxes: checkmark animado con path drawing
      \item Toggle switches: smooth slide transition (200ms)
      \item Number counters: animación conteo en cambio de valores grandes
    \end{itemize}
\end{itemize}

\subsubsection{Manejo de Datos Financieros}

\begin{itemize}
  \item \textbf{Visualización de montos}:
    \begin{itemize}
      \item Siempre incluir símbolo de moneda antes del monto
      \item Separadores de miles: coma (1,234.56)
      \item Ingresos en color verde, egresos en color rojo (solo el monto, no todo el card)
      \item Montos negativos con signo menos, no entre paréntesis
      \item Tamaño de fuente proporcional a importancia (monto total > monto individual)
    \end{itemize}
  
  \item \textbf{Gráficos y visualizaciones}:
    \begin{itemize}
      \item Estilo minimalista: ejes mínimos, sin grid excesivo
      \item Colores: usar paleta funcional, máximo 5 colores en un gráfico
      \item Labels claros y legibles
      \item Tooltips con información detallada on hover/tap
      \item Responsive: adaptar complejidad según tamaño de pantalla
    \end{itemize}
  
  \item \textbf{Input de montos}:
    \begin{itemize}
      \item Teclado numérico con decimales
      \item Preview del monto formateado en tiempo real
      \item Botones rápidos: +10, +50, +100, etc. (configurables)
    \end{itemize}
\end{itemize}

\subsection{Accesibilidad}

\begin{itemize}
  \item \textbf{Contraste de color}: WCAG AA mínimo, AAA preferido
  \item \textbf{Tamaño de texto}: Mínimo 16px para body, escalable hasta 200\%
  \item \textbf{Touch targets}: Mínimo 44x44px (iOS) / 48x48px (Android)
  \item \textbf{Screen readers}: 
    \begin{itemize}
      \item Todos los iconos interactivos con aria-label
      \item Headings semánticos (h1, h2, h3)
      \item Labels descriptivos en inputs
      \item ARIA roles apropiados (navigation, main, complementary)
    \end{itemize}
  \item \textbf{Navegación por teclado}: 
    \begin{itemize}
      \item Focus visible en todos los elementos interactivos
      \item Tab order lógico
      \item Shortcuts para acciones comunes
    \end{itemize}
  \item \textbf{Motion}: 
    \begin{itemize}
      \item Respetar \texttt{prefers-reduced-motion} del sistema
      \item Animaciones desactivables en settings
    \end{itemize}
\end{itemize}

\subsection{Responsive Design}

\begin{itemize}
  \item \textbf{Mobile First}: Diseñar primero para móvil, expandir a desktop
  \item \textbf{Breakpoints}:
    \begin{itemize}
      \item Mobile: < 640px (1 columna, bottom nav)
      \item Tablet: 640px - 1024px (2 columnas, sidebar opcional)
      \item Desktop: > 1024px (3 columnas, sidebar persistente)
    \end{itemize}
  \item \textbf{Adaptaciones por tamaño}:
    \begin{itemize}
      \item Mobile: FAB para acciones, menús bottom sheet
      \item Tablet: Más espacio horizontal, 2 paneles simultáneos
      \item Desktop: Dashboard complejo con múltiples widgets
    \end{itemize}
\end{itemize}

\subsection{Animaciones y Transiciones}

\begin{itemize}
  \item \textbf{Principios}:
    \begin{itemize}
      \item Rápidas y sutiles: 150-300ms máximo
      \item Propósito funcional, no decorativo
      \item Easing natural: ease-out para entrada, ease-in para salida
    \end{itemize}
  
  \item \textbf{Catálogo de animaciones}:
    \begin{itemize}
      \item \textit{Fade}: opacity 0 to 1, 200ms ease-out
      \item \textit{Slide}: transform translateY, 250ms ease-out
      \item \textit{Scale}: transform scale 0.95 to 1, 200ms ease-out
      \item \textit{Skeleton pulse}: shimmer effect con gradiente animado
    \end{itemize}
  
  \item \textbf{Cuándo usar}:
    \begin{itemize}
      \item Entrada/salida de elementos (modals, toasts)
      \item Cambio de estado (loading, success, error)
      \item Feedback de interacción (tap, hover)
      \item NO usar para navegación entre pantallas principales (demasiado frecuente)
    \end{itemize}
\end{itemize}

\subsection{Ejemplo de Pantalla: Dashboard Principal}

\textbf{Aplicación de principios minimalistas}:

\begin{enumerate}
  \item \textbf{Header limpio}:
    \begin{itemize}
      \item Fondo blanco puro, sin sombra
      \item Saludo: "Hola, [Nombre]" en H1 (24px, weight 600)
      \item Avatar pequeño (32px) a la derecha
      \item Sin decoraciones adicionales
    \end{itemize}
  
  \item \textbf{Card de resumen mensual} (hero card):
    \begin{itemize}
      \item Background: gradiente sutil gris claro a blanco
      \item Border-radius: 12px
      \item Padding generoso: 24px
      \item 3 números destacados alineados horizontalmente:
        \begin{itemize}
          \item Ingresos (verde, 18px, weight 600)
          \item Egresos (rojo, 18px, weight 600)
          \item Balance (azul, 24px, weight 700) - más grande
        \end{itemize}
      \item Labels en gris secundario (14px, weight 400) sobre cada número
      \item Divisores verticales sutiles entre números
    \end{itemize}
  
  \item \textbf{Gráfico de flujo de caja}:
    \begin{itemize}
      \item Line chart simple, sin grid de fondo
      \item Eje X: últimos 6 meses (labels en 12px, gris terciario)
      \item Eje Y: escala de montos (sin labels, solo valores en tooltips)
      \item Línea: 2px stroke, color brand
      \item Área bajo curva: gradiente sutil del color brand con opacidad 10\%
      \item Sin leyenda (solo una línea, obvio qué representa)
    \end{itemize}
  
  \item \textbf{Lista de transacciones recientes}:
    \begin{itemize}
      \item Título: "Recientes" (H3, 18px)
      \item 5 items, cada uno:
        \begin{itemize}
          \item Icono categoría (24px, color funcional)
          \item Descripción (16px, weight 400)
          \item Fecha (12px, gris terciario)
          \item Monto (16px, weight 600, color según tipo)
        \end{itemize}
      \item Separadores: línea gris ultra claro (1px) entre items
      \item Link al final: "Ver todas" (color brand, 14px)
    \end{itemize}
  
  \item \textbf{FAB}:
    \begin{itemize}
      \item Botón circular flotante, 56px diámetro
      \item Background color brand, icono + blanco
      \item Sombra: \texttt{box-shadow: 0 4px 16px rgba(59,130,246,0.3)}
      \item Posición: bottom-right, 16px margen
    \end{itemize}
  
  \item \textbf{Espaciado total}:
    \begin{itemize}
      \item 16px padding lateral en toda la pantalla
      \item 24px entre card de resumen y gráfico
      \item 24px entre gráfico y lista de transacciones
      \item 32px padding bottom para espacio del FAB
    \end{itemize}
\end{enumerate}

\subsection{Checklist de Diseño Minimalista}

Antes de implementar cada pantalla, validar:

\begin{enumerate}
  \item ¿Cada elemento tiene un propósito funcional claro?
  \item ¿Se puede eliminar algún elemento sin afectar funcionalidad?
  \item ¿El espacio en blanco guía la atención del usuario?
  \item ¿La jerarquía visual es obvia sin usar múltiples colores?
  \item ¿Los colores se usan solo para comunicar significado, no decoración?
  \item ¿Las fuentes son legibles en todos los tamaños de pantalla?
  \item ¿Los elementos interactivos son obvios sin necesidad de hover states?
  \item ¿Las animaciones mejoran la experiencia o solo agregan ruido?
  \item ¿La interfaz se siente rápida y responsive?
  \item ¿Cumple con estándares de accesibilidad (WCAG AA mínimo)?
\end{enumerate}

\end{document}